%% LyX 2.1.4 created this file.  For more info, see http://www.lyx.org/.
%% Do not edit unless you really know what you are doing.
\documentclass[english]{article}
\usepackage[T1]{fontenc}
\usepackage[latin9]{inputenc}
\usepackage{amstext}

\makeatletter

%%%%%%%%%%%%%%%%%%%%%%%%%%%%%% LyX specific LaTeX commands.
%% Because html converters don't know tabularnewline
\providecommand{\tabularnewline}{\\}

\makeatother

\usepackage{babel}
\begin{document}Z
\begin{table}
\begin{centering}
\begin{tabular}{lllll}
\hline 
Parameter  & value & SE & low 95\% CI & up 95\% CI\tabularnewline
\hline 
$b_{11}$ &  0.6837 & 0.1169 &  0.4546 & 0.9129\tabularnewline
$b_{21}$ &  0.2035 & 0.1045 & -0.0013 & 0.4084\tabularnewline
$b_{12}$ & -0.1999 & 0.1077 & -0.4111 & 0.0113\tabularnewline
$b_{22}$ &  0.7018 & 0.1044 &  0.4971 & 0.9065\tabularnewline
$\text{Mean winter temp}_{t+1}$ & \emph{-0.1466} & \emph{0.1210} & \emph{-0.3838} & \emph{0.0906}\tabularnewline
$\text{Winter precipitation}_{t+1}$ & \emph{-0.1720} & \emph{0.1185} & \emph{-0.4044} & \emph{0.0602}\tabularnewline
$\text{temperatureApril}_{t-4}$ &  0.2071 & 0.1062 & -0.0010 & 0.4153\tabularnewline
$\text{rainApril}_{t-4}$ & -0.0557 & 0.1068 & -0.2652 & 0.1537\tabularnewline
$\sigma_{1}^{2}$ &  0.3635 & 0.0742 &  0.2092 & 0.5602\tabularnewline
$\sigma_{2}^{2}$ &  0.3422 & 0.0734 &  0.1945 & 0.5314\tabularnewline
\hline 
\end{tabular}
\par\end{centering}

\caption{Coefficients for biotic and abiotic effects on population growth.
Species 1 is ptarmigan and species 2 gyrfalcon. Winter variables only
affect species 1 while April variables, delayed by 5 years (we model
the effect of variables at $t-4$ on growth between $t$ and $t+1$),
affect only species 2's population growth.  \emph{Effects of winter
variables are depicted in italics. } \label{tab:Estimates_MAR1both_winter}}
\end{table}


\begin{table}
\begin{centering}
\begin{tabular}{lllll}
\hline 
Parameter  & value & SE & low 95\% CI & up 95\% CI\tabularnewline
\hline 
$b_{11}$ &  0.6837 & 0.1169 &  0.4545 & 0.9129\tabularnewline
$b_{21}$ &  0.2035 & 0.1045 & -0.0013 & 0.4084\tabularnewline
$b_{12}$ & -0.1999 & 0.1078 & -0.4112 & 0.0113\tabularnewline
$b_{22}$ &  0.7018 & 0.1044 &  0.4971 & 0.9066\tabularnewline
$\text{Min Winter temp}_{t+1}$ & \emph{-0.1466} & \emph{0.1210} & \emph{-0.3838} & \emph{0.0906}\tabularnewline
$\text{Winter precipitation}_{t+1}$ & \emph{-0.1720} & \emph{0.1185} & \emph{-0.4044} & \emph{0.0603}\tabularnewline
$\text{temperatureApril}_{t-4}$ &  0.2071 & 0.1062 & -0.0010 & 0.4153\tabularnewline
$\text{rainApril}_{t-4}$ & -0.0557 & 0.1069 & -0.2654 & 0.1539\tabularnewline
$\sigma_{1}^{2}$ &  0.3635 & 0.0742 &  0.2092 & 0.5602\tabularnewline
$\sigma_{2}^{2}$ &  0.3422 & 0.0734 &  0.1945 & 0.5314\tabularnewline
\hline 
\end{tabular}
\par\end{centering}

\caption{Coefficients for biotic and abiotic effects on population growth.
Species 1 is ptarmigan and species 2 gyrfalcon. Winter variables only
affect species 1 while April variables, delayed by 5 years (we model
the effect of variables at $t-4$ on growth between $t$ and $t+1$),
affect only species 2's population growth. \emph{Effects of winter
variables are depicted in italics. }\label{tab:Estimates_MAR1both_winter2}}
\end{table}

\end{document}
